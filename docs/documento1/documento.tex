\documentclass{report}
\usepackage[T1]{fontenc} % Fontes T1
\usepackage[utf8]{inputenc} % Input UTF8
\usepackage[backend=biber, style=ieee]{biblatex} % para usar bibliografia
\usepackage{csquotes}
\usepackage[portuguese]{babel} %Usar língua inglesa
\usepackage{blindtext} % Gerar texto automáticamente
\usepackage[printonlyused]{acronym}
\usepackage{hyperref} % para autoref
\usepackage{graphicx}

\bibliography{bibliografia}


\begin{document}
%%
% Definições
%
\def\titulo{PROJETO 2 - PLANEAMENTO}
\def\data{17 de Maio de 2015}
\def\autores{Domingos Nunes, Dzianis Bartashevich, Francisco Cunha, Leonardo Oliveira}
\def\autorescontactos{dfsn@ua.pt, bartashevich@ua.pt, franciscomiguelcunha@ua.pt, leonardooliveira@ua.pt}
\def\versao{Versão 1.0}
\def\departamento{Departamento de Eletrónica, Telecomunicações e Informática}
\def\empresa{Universidade de Aveiro}
\def\logotipo{images/ua.pdf}
%
%% CAPA %%
%
\begin{titlepage}

\begin{center}
%
\vspace*{50mm}
%
{\Huge \titulo}\\ 
%
\vspace{10mm}
%
{\Large \empresa}\\
%
\vspace{10mm}
%
{\LARGE \autores}\\ 
%
%
\vspace{30mm}
%
\begin{figure}[h]
\center
\includegraphics{\logotipo}
\end{figure}
%
\vspace{20mm}
\end{center}
%
\begin{flushright}
\versao
\end{flushright}
\end{titlepage}

%
%
%%  Página de Título %%
%
%
\title{%
{\Huge\textbf{\titulo}}\\
{\Large \departamento\\ \empresa}
}
%
\author{%
    \autores \\
    \autorescontactos
}
%
\date{\data}
%
\maketitle
%%%%%%%%%%%%%%%%%%%%%%%%%%%%%%%%%%%%%%%%%%%
% RESUMO
%
%
\pagenumbering{roman}



\tableofcontents
\listoffigures


%%%%%%%%%%%%%%%%%%%%%%%%%%%%%%%
\clearpage
\pagenumbering{arabic}

%%%%%%%%%%%%%%%%%%%%%%%%%%%%%%%%
\chapter{Introdução}
\label{chap.introducao}
Este documento contém uma abordagem geral das etapas de realização do projeto 2 de Laboratórios de Informática, de acordo com o enunciado disponibilizado na plataforma \textit{moodle} da universidade \cite{moodle}.

Neste sentido, este documento está dividido em 7 capítulos. Depois desta introdução, no \autoref{chap.estrutura} é apresentada a estrutura das páginas \ac{html} necessárias à interação com a aplicação, no \autoref{chap.servidor} são referidos alguns aspetos sobre o servidor, no \autoref{chap.som}, são tratados os módulos necessários ao processamento e geração de ficheiros de som e no \autoref{chap.base} é apresentada a árvore da base de dados a ser criada. Já no \autoref{chap.fases}, estão presentes as datas planeadas para o desenvolvimento do projeto, assim como a distribuição de tarefas da fase inicial. Finalmente, no \autoref{chap.finais} são apresentadas algumas considerações finais.

%%%%%%%%%%%%%%%%%%%%%%%%%%%%%%%%
\chapter{Páginas Web}
\label{chap.estrutura}

bla bla

%%%%%%%%%%%%%%%%%%%%%%%%%%%%%%%%
\chapter{Servidor}
\label{chap.servidor}

bla bla

%%%%%%%%%%%%%%%%%%%%%%%%%%%%%%%%
\chapter{Som}
\label{chap.som}

Os módulos ligados ao processamento e geração de ficheiros de som seguirão os moldes do que foi sugerido no enunciado do projeto. Será criado um interpretador de pautas, um sintetizador e um processador de efeitos.

\chapter{Árvore da base de dados}
\label{chap.base}

A base de dados associada ao projecto serão duas tabelas \textbf{musics, interpretation} que estão interligadas entre si e as quais servirão para armazenar informação acerca dos conteúdos.
A tabela \textbf{musics} conterá informação com o \textbf{name} das músicas criadas e as \textbf{notes} (pautas) que lhes são correspondentes, sendo que a estes dois campos há um \textbf{id} que lhes identifique.
Por outro lado a tabela \textbf{interpretation} terá campos como
\textbf{\texttt{id\_music}} que estará associado à tabela \textbf{musics}. Conterá também campos como \textbf{(registration)} (registos) que servirá para a codificação do registo, que é um numero com oito dígitos,  e o campo \textbf{effects} (efeitos) servindo para guardar informação sobre e os efeitos utilizados na música.
Finalmente, os campos \textbf{upvotes} e \textbf{downvotes} guardarão respectivamente o número de votos positivos e negativos. 


A \autoref{tree} apresenta o esquema da árvore da base de dados.

\begin{figure}[htp]
\centering
\includegraphics[width=\textwidth]{images/tree.jpg}
\caption{Árvore da base de dados, com as tabelas dos utilizadores (\emph{users}), livros (\emph{books}) e requisições (\emph{requisitions}).}
\label{tree}
\end{figure}

%%%%%%%%%%%%%%%%%%%%%%%%%%%%%%%%
\chapter{Fases de desenvolvimento}
\label{chap.fases}
Estão previstas três fases de desenvolvimento do projeto:

\begin{itemize}
\item Fase 1 - 13 de Maio a 21 de Maio (a decorrer);
\item Fase 2 - 22 de Maio a 29 de Maio;
\item Fase 3 - 30 de Maio a 7 de Junho.
\end{itemize}

\section{Fase 1}
Inicialmente, será desenvolvida cada uma das partes da aplicação em separado e a distribuição das tarefas é a seguinte:
\begin{itemize}
\item Domingos - Implementação do \textit{mockup} das páginas \ac{html};
\item Dzianis - Implementação do servido;
\item Francisco - Implementação da base de dados;
\item Leonardo - Implementação do sintetizador e interpretador de pautas.
\end{itemize}
\section{Fase 2}
Neste fase a aplicação será testada no seu todo, sendo efetuados os ajustes necessários ao seu correto funcionamento, podendo ser acrescentadas algumas funcionalidades úteis que não tenham sido previstas.
\section{Fase 3}
Finalmente, serão afetuados os ajustes finais, sendo também completado o módulo processador de efeitos, para que este comece efetivamente a criar efeitos nas músicas: Echo, Tremolo, Distorção, Percursão, Chorus e Envelope.

%%%%%%%%%%%%%%%%%%%%%%%%%%%%%%%%
\chapter{Considerações finais}
\label{chap.finais}

As considerações feitas neste documento estão abertas a alterações, pelo que poderão ser efetuados ajustes nas funcionalidades implementadas ou na distribuição de tarefas caso seja necessário.

Embora não tenha sido aqui referido com pormenor, serão efetuados alguns testes (manuais e automáticos) a cada um dos componentes desenvolvidos e que serão, sempre que possível, da responsabilidade de um elemento que não tenha desenvolvido esse componente, para que seja mais ágil a deteção de erros e para que todos possamos ter uma melhor perseção do que foi feito pelos restantes colegas.


%%%%%%%%%%%%%%%%%%%%%%%%%%%%%%%%%
\chapter*{Acrónimos}
\begin{acronym}
\acro{sql}[SQL] {Structured Query Language}
\acro{html} [HTML] {HyperText Markup Language}

\end{acronym}


%%%%%%%%%%%%%%%%%%%%%%%%%%%%%%%%%
\printbibliography

\end{document}
